\documentclass[10pt,letterpaper,titlepage]{article}
\usepackage{titletoc}

\usepackage{ifthen,varioref}
\usepackage{url}
\usepackage{titlesec}
\usepackage{fullpage}
\usepackage{ulem}
\usepackage{hyperref}

% fix block paragraphs in toc
\usepackage{parskip}
% use block paragraphs
\setlength{\parindent}{0pt}
%\setlength{\parskip}{\baselineskip}
\setlength{\parskip}{4pt}

\newcommand\corpname{Gainesville Hackerspace, Inc.}
\newcommand\approvaldate{March 9, 2010}

\hypersetup{
    pdfauthor={Ian Taylor},
    pdftitle={\corpname{} Bylaws},
    pdfborder={0 0 0},
    pdfsubject={bylaws},
    pdfkeywords={bylaws, \corpname},
    linkcolor=blue,
    citecolor=blue,
    urlcolor=blue,
    bookmarks=true,
    colorlinks=false
}

\titlecontents{section}[1.5em]
    {}
    {\contentslabel{2.0em}}
    {\hspace*{-2.0em}}
    {\titlerule*[1pc]{.}\contentspage}

\renewcommand\thesection{\Roman{section}}
\renewcommand\thesubsection{\arabic{subsection}}
\titleformat{\section}{\bfseries\Large}{Article~\thesection}{0.5em}{}
\titleformat{\subsection}{\bfseries\large}{Section~\thesubsection}{0.5em}{}
\labelformat{section}{Article~#1}
\labelformat{subsection}{\protect\iscurrentsection{\thesection}Section~#1}
\newcommand\iscurrentsection[1]{%
    \ifthenelse{\equal{#1}{\thesection}}{}{Article~#1~}}

% for things like 1\st 2\nd 3\rd 4\th
\newcommand{\superscript}[1]{\ensuremath{^{\textrm{#1}}}}
\newcommand{\subscript}[1]{\ensuremath{_{\textrm{#1}}}}
\renewcommand{\th}[0]{\superscript{th}}
\newcommand{\st}[0]{\superscript{st}}
\newcommand{\nd}[0]{\superscript{nd}}
\newcommand{\rd}[0]{\superscript{rd}}

\title{%
    \textsc{\LARGE \corpname \\
    \Large Bylaws
    }
}
\date{\today}

\begin{document}
\maketitle

\tableofcontents
\newpage

\section{Why We Exist}
\label{whyweexist}

By law, this is the core governing document of \corpname

Amendments are handled as specified in \ref{amendments}

These Bylaws were approved by the Incorporators on \approvaldate

\subsection{General Purposes}
\label{genpurp}

Said corporation is organized exclusively for charitable, educational, and
scientific purposes within the meaning of Section 501(c)(3) of the Internal
Revenue Code, or the corresponding subsection of any future federal tax code.
The mission of the corporation is to improve the world by creatively rethinking
technology. 

\subsection{Specific Purposes}

Subject to and within the limits of \ref{genpurp}, the corporation shall
\begin{itemize}
    \item Build or lease and maintain spaces suitable for technical, artistic
          and social collaboration.
    \item Collaborate on all forms of technology, culture and craft in new and
          interesting ways.
    \item Apply the results of its work to cultural, charitable and
          scientific causes.
    \item Share its research and discoveries, using what is learned to
          teach others.
    \item Recruit and develop members to participate in these activities.
\end{itemize}

\section{Who We Are}

\subsection{Designation of Membership Classes}
 
\corpname{} has a voting membership class and a non-voting membership class. 

\subsection{Membership Class Qualifications}
 
Any person who supports the purposes laid out in \ref{whyweexist} of these
bylaws is qualified to become a member.

\subsection{Voting Membership Class Election}
 
Any member may nominate a qualified person to be a voting member.  Any eligible
person may be elected as a voting member at any regular meeting upon payment of
their first periodic dues and visual approval of 51\% of voting members
present.
For purposes of these bylaws, all persons listed as initial directors on the
Articles of Incorporation shall be considered the initial voting members.

\subsection{Voting Membership Dues}
\label{votmemdue}
 
The amount, payment period, due date and acceptable methods for collection of
dues shall be reviewed each year at the annual meeting.
A majority vote of the members at the annual meeting may change the procedure
and terms for payment of dues.
A member may request that their membership be put in suspension for a maximum
of 3 months.
Member dues are owed on or before the 15\th of every month.

\subsection{Voting Membership Rights and Responsibilities}
 
Each voting member shall have an equal right to voice their opinion and vote
their preference or abstain from voting in the affairs of the corporation.
Each voting member shall exercise only one vote for each decision before the
corporation.
Each voting member shall have reasonable inspection rights of corporate
records.
Each voting member shall be responsible for timely payment of dues, providing
their current address, contact information, and preference for electronic
receipt of communications.
Each voting member is responsible for continuing to support the purposes of the
corporation.

Each voting member shall have the right to access the space 24 hours a day, 7
days a week.
Each voting member shall have the right to use equipment and supplies purchased
by the corporation for purposes subject to and within the limits of
\ref{whyweexist}. 

\subsection{Voting Membership Resignation and Termination}
 
Any voting member may resign by filing a written resignation with the
Secretary.
Resignation shall not relieve a voting member of unpaid dues or other monies
owed.
Voting membership may be suspended for non-payment of dues by the Treasurer.
Any suspended voting member may restore their membership 90 days after
suspension upon payment of dues owed and payable through one month beyond the
end of the suspension period, or upon the granting of a dues waiver as outlined
in \ref{votmemdue}.
Voting membership may also be terminated for any reason by written petition
signed by more than three quarters ($\frac{3}{4}$) of the voting members.

\subsection{Non-Voting Membership}
 
The non-voting member class exists for fundraising, voting member recruitment,
and honorary purposes.
Non-voting memberships and membership titles are subject to voting member
approval.
Non-voting members do not have the right to vote in affairs of the corporation
nor do they have any responsibilities towards it.
Non-voting members have the right to be on the premises only when accompanied
and supervised by a voting member.
Non-voting members do not have the right to use equipment and supplies
purchased by the corporation for self-directed projects.
Non-voting members may only use these resources for the purposes of group
projects under the supervision of a voting member.
All other rights and responsibilities of non-voting members shall be explicitly
stated by the Treasurer and subject to voting member approval.

\section{Meetings}

\subsection{Regular Meetings}
 
Regular meetings of voting members shall be held every Tuesday at 19:30 local
time at the registered office.
A different meeting place may be designated by written petition signed by more
than three quarters ($\frac{3}{4}$) of voting members.
Regular meetings shall not take place on the day before, upon, or after a
federal holiday unless specifically approved at the prior regular meeting or
annual meeting.
Meetings shall not take place on the day of a closure declared by the Office of
Personnel Management. 

\subsection{Annual Meetings}
 
An annual meeting of all members shall take place sometime in January, February
or March.
The Board of Directors shall select the date, time and place no later than
January 31 of each year.
The date, time and place of the annual meeting must be posted in the registered
office and submitted to members electronically at least two weeks prior to the
annual meeting.
A petition signed by more than three quarters ($\frac{3}{4}$) of voting members
and submitted to the Board of Directors before President’s Day may specify a
new date, time and place for the annual meeting.
At the annual meeting, the voting members shall elect the Board of Directors,
review and vote on the standing rules and policies of the corporation, receive
reports on the activities of the corporation, approve the budget and determine
the direction of corporation in the coming year. 

\subsection{Special Meetings}
\label{specialmeet}
 
A petition presented to all voting members and signed by one third
($\frac{1}{3}$) of voting members may call a special meeting.
Such a petition must include the date, time, place and agenda of the special
meeting. 

\subsection{Notice of Meetings}
 
The time and place of upcoming meetings shall be conspicuously posted at the
registered office and electronically sent to all voting members.
No notice is required for a regular meeting.
Special meetings require 72 hours notice considered delivered only when all
voting members are personally notified and given an opportunity to sign a
special meeting petition.
The agenda of the next upcoming meeting and minutes of the previous meeting
shall be posted at the registered office and electronically submitted to all
voting members at least 72 hours prior to any meeting. 

\subsection{Quorum}
 
At a duly called meeting, at least 50\% ($\frac{1}{2}$) of the entire voting
membership shall constitute a quorum. 

\subsection{Voting}
 
When a quorum is present, all issues, except when otherwise specified in these
bylaws, shall be decided by affirmative vote of more than 50\% ($\frac{1}{2}$)
of the voting members present. 

\subsection{Conduct of Meetings}
\label{condofmeet}
 
All meetings shall follow the Robert's Rules of Order as approved at an annual
meeting.

\section{The Officers}
\label{officers}

\subsection{Role, Number, Qualification, Term and Compensation}
 
There shall be five officers consisting of President, Vice President,
Secretary, Treasurer, and Sergeant at Arms.
Each officer must be a director and each officer shall serve from the time of
their election until their successor is elected and qualifies.
No officer may serve more than 3 consecutive terms.
A term shall begin at the conclusion of the annual meeting as defined in
\ref{specialmeet} and shall last until the next annual meeting.
No officer shall be compensated for their service as an officer, though the
corporation may provide insurance and indemnity for officers as allowed by law.

\subsection{Duties of the President}

The President shall preside over all meetings or designate an alternate,
attempt to achieve consensus in all decision-making, ensure the membership is
informed of all relevant issues, and serve other duties of a President as
required by law or custom. 

\subsection{Duties of the Vice President}

The Vice President shall be primarily responsible for the information systems
and communication processes of the corporation, coordinate the teams that
manage and design those systems, draft policies and procedures for information
system use, ensure effective communication and information exchange within the
corporation, and serving all other duties of a Vice President as required by
law or custom, including acting when the President is unable or unwilling to
act.

\subsection{Duties of the Secretary}

The Secretary shall be responsible for membership records including membership
and board meeting actions and petitions, sending out meeting announcements,
posting and distributing copies of membership meeting minutes and relevant
meeting agendas to the membership, assuring that corporate records are
maintained, and serving all other duties of a Secretary as required by law or
custom. 

\subsection{Duties of the Treasurer}

The Treasurer shall be custodian of corporate funds, collect dues, make a
financial report for each meeting, assist in the preparation of the budget,
develop fundraising plans, make financial information available to members and
the public, sneer at members who do not pay dues in a timely fashion and serve
all other duties of a Treasurer as required by law or custom. 

\subsection{Duties of the Sergeant at Arms}

The Sergeant at Arms keeps order at all meetings.
Makes sure that everyone upholds the rules outlined in \ref{condofmeet} during
all meetings.

\subsection{Duties of the Officers as whole to provide an Annual Report}
 
The Officers must prepare an annual report to be distributed at the annual
meeting.
The report should chronicle the activities of the corporation, including
specific narratives on the corporation's work, the corporation's annual
financial statements, relevant legal filings, and relevant copies of the
organization's district and federal tax returns.

\subsection{Officers are Directors}
 
The officers shall also serve as bona-fide directors on the Board of Directors.
Election, resignation, removal and vacancies of the officers are handled in
accordance with procedures laid out in \ref{boardofdirect}.

\section{The Board of Directors}
\label{boardofdirect}

\subsection{Role, Size, Term and Compensation}
 
The Board of Directors shall consist of the five officers as defined in
\ref{officers}, all of whom are considered directors for the purposes of this
article.
All directors must be voting members of the corporation.
Each director shall serve from the time of their election until their successor
is elected and qualifies.
No member may serve more than 3 consecutive terms on the board of directors.
No director may be compensated for their service as a board member, though the
corporation may provide insurance and indemnity for board members as allowed by
law.

\subsection{Meetings}
 
The board of directors shall meet when necessary, provided all voting members
receive notice sent electronically at least five business days prior to the
meeting.
All voting members may attend a meeting of the board of directors.
The notice shall give the time, place, reason for calling the meeting and the
agenda for said meeting.
Notices shall be conspicuously posted at the registered office and
electronically distributed to all members at least five business days prior to
a meeting.
Minutes shall follow the standing rules for meetings as approved at an annual
meeting.
Minutes of each board meeting shall be conspicuously posted at the registered
office and electronically distributed to members within 48 hours.
Minutes shall be considered approved when signed by all board members in
attendance.

\subsection{Elections}
\label{elections}
 
Each member present shall be given an opportunity to be a candidate for each
officer position.
If there is more than one candidate for an officer position, the candidate
which obtains the highest number votes from voting members present shall be
elected.
If there are no candidates for an officer position, the outgoing officeholder
may, if eligible, elect to serve another term or select any willing member to
serve in that position.

\subsection{Quorum}
 
Two thirds of board of directors at a duly called board member meeting shall
constitute a quorum. 

\subsection{Voting}
 
All issues, except when otherwise specified in these bylaws, shall be decided
by affirmative vote of more than half of the directors present at a duly held
meeting. 

\subsection{Resignation, Termination and Vacancies}
 
Any officer or director may resign by filing a written resignation with the
Secretary or two other board members.
A Director may be removed from office only by unanimous vote of all directors
excluding the person to be removed.
Once a vote has taken place to remove a member of the board of directors no
further votes may occur to remove the director voted on until the following
annual meeting.
Vacancies on the board shall be filled at the next regular meeting using the
applicable process outlined in \ref{elections}.

Directors removed by vote of directors may qualify to run for office again
immediately after being removed from office.

\section{Committees}

\subsection{Committee formation}

The board may create committees as needed, such as fundraising, facilities,
public relations, data collection, etc.
The board chair appoints all committee chairs.

\subsection{Finance Committee}

The treasurer is the chair of the Finance Committee, which may include three
other board members.
The Finance Committee is responsible for developing and reviewing fiscal
procedures, fundraising plans, and the annual budget with Member volunteers and
other Board Members.
The board must approve the budget and all expenditures must be within budget.
Any major change in the budget must be approved by the board.
The fiscal year shall be the calendar year.
Annual reports are required to be submitted to the board showing income,
expenditures, and pending income.
The financial records of the organization are public information and shall be
made available to the membership, board members, and the public.

\subsection{Auditing Committee}

On or before September 30, the members shall nominate and approve an audit
committee consisting of three voting members who are not members of the board
of directors and have not served as a director for 180 days prior to
appointment.
The audit committee shall have full inspection rights to the affairs and
documentation of the corporation.
No audit committee shall be convened if the corporation has fewer than nine
voting members.
Otherwise, the board or voting members may not create committees or delegate
their authority. 

\section{Amendments}
\label{amendments}
 
These bylaws may be amended only when an amendment proposal petition is
approved at a membership meeting and signed by more than three quarters
($\frac{3}{4}$) of voting members.
Written notice of such petition must be submitted electronically to all members
and delivered to all members of the corporation physically in person or by
registered mail to take effect.

\end{document}
